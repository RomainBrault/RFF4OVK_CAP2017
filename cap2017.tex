\documentclass[twocolumn]{article}

\usepackage[width=17cm,height=22cm]{geometry}
\usepackage[french]{babel}
\usepackage[utf8]{inputenc}
\usepackage{fancyvrb}
\usepackage{authblk}
\usepackage{hyperref}
\usepackage{pythontex}

\newtheorem{theorem}{Théorème}

\bibliographystyle{alpha}

\title{Exemple d'article au format CAp'2017}
\author[1]{Auteur A\thanks{Si vraiment vous voulez mettre votre email
    ou page web...: A.A@univ-x.fr}}
\author[2]{Auteur B}
\affil[1]{Université X, CNRS}
\affil[2]{Université Y, CNRS et INRIA}

\renewcommand\Authands{, et }
\renewcommand\Authand{ et }
\begin{document}
\maketitle

\begin{abstract}
  Ce style est très simple. Vous pouvez ajouter des mots-clef sous le
  résumé. Le nombre de pages est limité à 10. 
\end{abstract}

\medskip

\noindent\textbf{Mots-clef}: Style, \LaTeX, CAP.

\section{L'entête}
\label{sec:lentete}

Rien de spécial. Ajoutez l'option \texttt{twocolumn} à la classe
\texttt{article}. Le paquet \texttt{geometry} permet facilement de
régler les tailles de papier:

\begin{Verbatim}
\documentclass[twocolumn]{article}
\usepackage[width=17cm,height=24cm]{geometry}
\end{Verbatim}

Pour les affiliations, vous pouvez utiliser
\href{http://ctan.org/pkg/authblk}{le paquet \texttt{authblk}}.


Et évidemment, vous ajoutez ensuite les paquets que vous voulez
utiliser, les macros les définitions de théorèmes etc... Nous
recommandons le paquet \texttt{hyperref} puisque les documents
\texttt{PDF} seront en ligne si vous avez donné votre accord.

\begin{Verbatim}
  \usepackage[french]{babel}
  \usepackage[utf8]{inputenc}
  \usepackage{fancyvrb}
  \usepackage{authblk}
  \usepackage{amsfonts} 
  \usepackage{amssymb}
  \usepackage{amsthm,amsmath} 
  \usepackage{hyperref}
\end{Verbatim}

\begin{Verbatim}
  \newtheorem{definition}{Définition} 
  \newtheorem{theorem}{Théorème}
  \newtheorem{lemma}{Lemme} 
  \newtheorem{proposition}{Proposition}
  \newtheorem{corollary}{Corollaire}
\end{Verbatim}

\section{Le corps}
\label{sec:le-corps}

\begin{theorem}
  Ceci est un beau théorème.
\end{theorem}

Bla bla bla bla bla bla bla bla bla bla bla bla bla bla bla bla bla
bla bla bla bla bla bla bla bla bla bla bla bla bla bla bla bla bla
bla bla bla bla bla bla bla bla bla bla bla bla bla bla bla bla bla
bla bla bla bla bla bla bla bla bla bla bla bla bla bla bla bla bla
bla bla bla bla bla bla bla bla bla bla bla bla bla bla bla bla bla
bla bla bla bla bla bla bla bla bla bla bla bla bla bla bla bla bla
bla bla bla bla bla bla bla bla bla bla



\section{Les figures}
\label{sec:les-figures}
Utilisez les formes avec \texttt{*} pour mettre une table ou figure
sur deux colonnes (voir la table~\ref{tab:tab} ou la
figure~\ref{fig:code}). Bien sûr les formes sans étoile restent
utlisables pour une seule colonne comme dans la table~\ref{tab:tabcol}.

\begin{table*}[htbp]
  \centering
  \begin{tabular}{|l|l|l|l|}
    \hline
    \textbf{Colonne 1} & \textbf{Colonne 2} & \textbf{Colonne 3} &
    \textbf{Colonne 4} \\ \hline
    1 & 2 & 3 & 4 \\ 
    1 & 2 & 3 & 4 \\ 
    1 & 2 & 3 & 4 \\ 
    1 & 2 & 3 & 4 \\ \hline
  \end{tabular}
  \caption{Une grande table}
  \label{tab:tab}
\end{table*}

\begin{table}[htbp]
  \centering
  \begin{tabular}{|l|l|}
    \hline
    \textbf{Colonne 1} & \textbf{Colonne 2}  \\ \hline
    1 & 2 \\ 
    1 & 2 \\ 
    1 & 2 \\ 
    1 & 2 \\ \hline
  \end{tabular}
  \caption{Une petite table}
  \label{tab:tabcol}
\end{table}

\begin{SaveVerbatim}{codefigure}
  \begin{table*}[t]
    \centering
    \begin{tabular}{|l|l|l|l|}
      \hline
      \textbf{Colonne 1} & \textbf{Colonne 2} & \textbf{Colonne 3} &
      \textbf{Colonne 4} \\ \hline
      1 & 2 & 3 & 4 \\ 
      1 & 2 & 3 & 4 \\ 
      1 & 2 & 3 & 4 \\ 
      1 & 2 & 3 & 4 \\ \hline
    \end{tabular}
    \caption{Une grande table}
    \label{tab:tab}
  \end{table*}
\end{SaveVerbatim}


\begin{figure*}[t]
  \centering
  \UseVerbatim[frame=single]{codefigure}
  \caption{Un exemple de code sur deux colonnes.}
  \label{fig:code}
\end{figure*}
Bla bla bla bla bla bla bla bla bla bla bla bla bla bla bla bla bla
bla bla bla bla bla bla bla bla bla bla bla bla bla bla bla bla bla
bla bla bla bla bla bla bla bla bla bla bla bla bla bla bla bla bla
bla bla bla bla bla bla bla bla bla bla bla bla bla bla bla bla bla
bla bla bla bla bla bla bla bla bla bla bla bla bla bla bla bla bla
bla bla bla bla bla bla bla bla bla bla bla bla bla bla bla bla bla
bla bla bla bla bla bla bla bla bla bla



\section{La bibliographie}
\label{sec:la-bibliographie}

Rien à dire. C'est comme par défaut. Par exemple~\cite{papier1} et~\cite{papier2}.

Bla bla bla bla bla bla bla bla bla bla bla bla bla bla bla bla bla
bla bla bla bla bla bla bla bla bla bla bla bla bla bla bla bla bla
bla bla bla bla bla bla bla bla bla bla bla bla bla bla bla bla bla
bla bla bla bla bla bla bla bla bla bla bla bla bla bla bla bla bla
bla bla bla bla bla bla bla bla bla bla bla bla bla bla bla bla bla
bla bla bla bla bla bla bla bla bla bla bla bla bla bla bla bla bla
bla bla bla bla bla bla bla bla bla bla

\section{Sans \LaTeX}
\label{sec:sans-latex}

Si vous n'utilisez pas \LaTeX, vous pouvez reprendre les dimensions
données au début de ce document et essayer de trouver une fonte
lisible. 

\bibliography{cap2017}

\end{document}
